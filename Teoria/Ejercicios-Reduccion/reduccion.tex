\documentclass[11pt,a4paper]{article}
\usepackage[spanish,es-nodecimaldot]{babel}	% Utilizar español
\usepackage[utf8]{inputenc}					% Caracteres UTF-8
\usepackage{graphicx}						% Imagenes
\usepackage[hidelinks]{hyperref}			% Poner enlaces sin marcarlos en rojo
\usepackage{fancyhdr}						% Modificar encabezados y pies de pagina
\usepackage{float}							% Insertar figuras
\usepackage[textwidth=390pt]{geometry}		% Anchura de la pagina
\usepackage[nottoc]{tocbibind}				% Referencias (no incluir num pagina indice en Indice)
\usepackage{enumitem}						% Permitir enumerate con distintos simbolos
\usepackage[T1]{fontenc}					% Usar textsc en sections
\usepackage{amsmath}						% Símbolos matemáticos

\usepackage{listings}
\usepackage{xcolor}

\definecolor{codegreen}{rgb}{0,0.6,0}
\definecolor{codegray}{rgb}{0.5,0.5,0.5}
\definecolor{codepurple}{rgb}{0.58,0,0.82}
\definecolor{backcolour}{rgb}{0.95,0.95,0.92}

\lstdefinestyle{mystyle}{
    backgroundcolor=\color{backcolour},   
    commentstyle=\color{codegreen},
    keywordstyle=\color{magenta},
    numberstyle=\tiny\color{codegray},
    stringstyle=\color{codepurple},
    basicstyle=\ttfamily\footnotesize,
    breakatwhitespace=false,         
    breaklines=true,                 
    captionpos=b,                    
    keepspaces=true,                 
    numbers=left,                    
    numbersep=5pt,                  
    showspaces=false,                
    showstringspaces=false,
    showtabs=false,                  
    tabsize=4,
    language=C++
}

\lstset{style=mystyle}


% Comando para poner el nombre de la asignatura
\newcommand{\asignatura}{Arquitectura y Computación de Altas Prestaciones}
\newcommand{\autor}{Vladislav Nikolov Vasilev}
\newcommand{\titulo}{Ejercicios de teoría}
\newcommand{\subtitulo}{Reducción}
\newcommand{\rama}{Ingeniería de Computadores}

% Configuracion de encabezados y pies de pagina
\pagestyle{fancy}
\lhead{\autor{}}
\rhead{\asignatura{}}
\lfoot{Grado en Ingeniería Informática}
\cfoot{}
\rfoot{\thepage}
\renewcommand{\headrulewidth}{0.4pt}		% Linea cabeza de pagina
\renewcommand{\footrulewidth}{0.4pt}		% Linea pie de pagina

\begin{document}
\pagenumbering{gobble}

% Pagina de titulo
\begin{titlepage}

\begin{minipage}{\textwidth}

\centering

%\includegraphics[scale=0.5]{img/ugr.png}\\
\includegraphics[scale=0.3]{img/logo_ugr.jpg}\\[1cm]

\textsc{\Large \asignatura{}\\[0.2cm]}
\textsc{GRADO EN INGENIERÍA INFORMÁTICA}\\[1cm]

\noindent\rule[-1ex]{\textwidth}{1pt}\\[1.5ex]
\textsc{{\Huge \titulo\\[0.5ex]}}
\textsc{{\Large \subtitulo\\}}
\noindent\rule[-1ex]{\textwidth}{2pt}\\[3.5ex]

\end{minipage}

%\vspace{0.5cm}
\vspace{0.7cm}

\begin{minipage}{\textwidth}

\centering

\textbf{Autor}\\ {\autor{}}\\[2.5ex]
\textbf{Rama}\\ {\rama}\\[2.5ex]
\vspace{0.3cm}

\includegraphics[scale=0.3]{img/etsiit.jpeg}

\vspace{0.7cm}
\textsc{Escuela Técnica Superior de Ingenierías Informática y de Telecomunicación}\\
\vspace{1cm}
\textsc{Curso 2019-2020}
\end{minipage}
\end{titlepage}

\pagenumbering{arabic}
\tableofcontents
\thispagestyle{empty}				% No usar estilo en la pagina de indice

\newpage

\setlength{\parskip}{1em}

\section{Ejercicio transparencia 23}

En este ejercicio se pide determinar qué elementos tendría la \textbf{hebra 20 del bloque 1}
teniendo en cuenta que se tiene un vector de 256 elementos y 2 bloques con 64 hebras cada
uno. A continuación se proporciona el código comentado:

\begin{lstlisting}
unsigned int dimHebras = 128 / 2; // dimHebras = 64

// sumaParcial[2*64] -> sumaParcial[128]
__ shared__ float sumaParcial[2*dimHebras];

unsigned int h = id_hebra; // h = 20

// start = 2 * 1 * 64 = 128
unsigned int start=2*id_bloque*dimHebras;

// sumaParcial[20] = Input[128 + 20] -> sumaParcial[20] = Input[148]
sumaParcial[h]=Input[start+h];

// sumaParcial[64+20]=Input[128+64+20] -> sumaParcial[84]=Input[212]
sumaParcial[N+h]=Input[start+dimHebras+h];
\end{lstlisting}

\section{Ejercicio transparencia 22: versión modificada}

Se ha proporcionado una versión modificada del ejemplo que se puede ver en la
transparencia 22. Aquí se tiene un \textbf{vector de entrada de 32 elementos}, los cuáles
se reparten entre \textbf{2 bloques con 8 hebras cada uno}. Se ha calculado qué elementos
del vector de entrada tendrían las dos primeras hebras de cada bloque.

\subsection{Hebra 0 del bloque 0}

\begin{lstlisting}
unsigned int dimHebras = 16 / 2; // dimHebras = 8

// sumaParcial[2*8] -> sumaParcial[16]
__ shared__ float sumaParcial[2*dimHebras];

unsigned int h = id_hebra; // h = 0

// start = 2 * 0 * 8 = 0
unsigned int start=2*id_bloque*dimHebras;

// sumaParcial[0] = Input[0 + 0] -> sumaParcial[0] = Input[0]
sumaParcial[h]=Input[start+h];

// sumaParcial[8+0]=Input[0+8+0] -> sumaParcial[8] = Input[8]
sumaParcial[N+h]=Input[start+dimHebras+h];
\end{lstlisting}

\subsection{Hebra 1 del bloque 0}

\begin{lstlisting}
unsigned int dimHebras = 16 / 2; // dimHebras = 8

// sumaParcial[2*8] -> sumaParcial[16]
__ shared__ float sumaParcial[2*dimHebras];

unsigned int h = id_hebra; // h = 1

// start = 2 * 0 * 8 = 0
unsigned int start=2*id_bloque*dimHebras;

// sumaParcial[1] = Input[0 + 1] -> sumaParcial[1] = Input[1]
sumaParcial[h]=Input[start+h];

// sumaParcial[8+1]=Input[0+8+1] -> sumaParcial[9] = Input[9]
sumaParcial[N+h]=Input[start+dimHebras+h];
\end{lstlisting}

\subsection{Hebra 0 del bloque 1}

\begin{lstlisting}
unsigned int dimHebras = 16 / 2; // dimHebras = 8

// sumaParcial[2*8] -> sumaParcial[16]
__ shared__ float sumaParcial[2*dimHebras];

unsigned int h = id_hebra; // h = 0

// start = 2 * 1 * 8 = 16
unsigned int start=2*id_bloque*dimHebras;

// sumaParcial[0] = Input[16 + 0] -> sumaParcial[0] = Input[16]
sumaParcial[h]=Input[start+h];

// sumaParcial[8+0]=Input[16+8+0] -> sumaParcial[8] = Input[24]
sumaParcial[N+h]=Input[start+dimHebras+h];
\end{lstlisting}

\subsection{Hebra 1 del bloque 0}

\begin{lstlisting}
unsigned int dimHebras = 16 / 2; // dimHebras = 8

// sumaParcial[2*8] -> sumaParcial[16]
__ shared__ float sumaParcial[2*dimHebras];

unsigned int h = id_hebra; // h = 1

// start = 2 * 1 * 8 = 16
unsigned int start=2*id_bloque*dimHebras;

// sumaParcial[0] = Input[16 + 1] -> sumaParcial[1] = Input[16]
sumaParcial[h]=Input[start+h];

// sumaParcial[8+1]=Input[16+8+1] -> sumaParcial[9] = Input[25]
sumaParcial[N+h]=Input[start+dimHebras+h];
\end{lstlisting}

\end{document}

